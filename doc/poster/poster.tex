\documentclass[a4paper]{jpconf}
\usepackage{graphicx}
\begin{document}
\title{\textsc{MaGe}: a Monte Carlo framework for the Gerda and Majorana 
double beta decay experiments}

\author{M Bauer, S Belogurov, YD Chan, M Descovich, J Detwiler, M Di Marco, 
B Fujikawa, D Franco, V Gehman, R Henning, K Hudek, R Johnson, D Jordan, 
K Kazkaz, A Klimenko, M Knapp, K Kroeninger, K Lesko, X Liu, M Marino, 
A Mokhtarani, L Pandola$^1$, M Perry, A Poon, D Radford, C Tomei and C Tull}

%\address{$^1$ INFN, LNGS, S.S. 17/bis km18+910, 67010 Assergi (AQ). Italy}
\ead{$^1$luciano.pandola@lngs.infn.it}
%
\begin{abstract}
The Gerda~\cite{gerda} and Majorana~\cite{majo} projects, both searching for 
the neutrinoless double beta-decay of $^{76}$Ge, are developing a joint 
Monte-Carlo simulation framework called \textsc{MaGe}. Such an approach has 
many benefits: the workload for the development of general tools is shared 
between more experts, the code is tested in more detail, and more experimental 
data is made available for validation. 
%The flexibility of the Object Oriented (OO) interface also allows each 
%collaboration to develop packages specific to their application independently.
%\textsc{MaGe} is already being used for the design of the Gerda and Majorana 
%experiments. 
\end{abstract}
%
\textsc{MaGe} is a \textsc{GEANT4}-based simulation package maintained and 
developed by a joint group of the Gerda~\cite{gerda} and Majorana~\cite{majo} 
collaborations. Since both search for $0\nu 2\beta$ decay in $^{76}$Ge, many 
common tools, like event generators and physics processes, can be shared; more 
users test the package, and the simulation is validated with experimental data 
coming from independent measurements. The object-oriented coding technique 
makes the \textsc{MaGe} package flexible and versatile, supporting different 
detector geometries and output schemes (like \textsc{Root}).  
\textsc{Geant4} was selected as the underlying Monte-Carlo toolkit because 
(1) it provides the full simulation chain, from the event generator to the 
output and visualization, (2) it is now well-established in the Particle 
Physics community and (3) it includes a wide set of physics models. 
%On the other hand, the OO structure of 
%\textsc{Geant4} is very suitable for the development of a common framework 
%considering the geographical distribution of the groups partecipating 
%in \textsc{MaGe}. 
The default physics list includes specific low-energy electromagnetic
models (including atomic effects, like x-ray  
fluorescence) and general-purpose hadronic processes, describing in particular 
muon spallation, neutron tracking and isotope production. Simulation
of the electric fields and pulse-shape analysis, and models to
handle the optical photons, generated by scintillation or \v{C}erenkov effect, 
are also being developed.  
The main design principle for a $2\beta$ decay application is the
reduction of backgrounds. The numerous sources (radioactive chains,
cosmic rays, etc) were studied with \textsc{MaGe} for the
design of the $^{76}$Ge target and electronics, but also for the
supporting structures and 
surrounding materials. 
%\textsc{MaGe} allows to select specific implementation (geometry, physics, 
%generators, i/o) at run-time without recompiling the code, making possible 
%for new users to readily use \textsc{MaGe} and produce results without 
%knowing in details the class structure and C++ programming. 
\ack This work was supported in part by the U.S. Department of 
Energy under Contract No. DE-AC02-05CH11231 .
\section*{References}
\begin{thebibliography}{9}
\bibitem{gerda} Gerda Collaboration, Proposal, at
http://www.mpi-hd.mpg.de/ge76/home.html
\bibitem{majo} Majorana Collaboration, {\it Preprint} nucl-ex/0311013 
\end{thebibliography}



\end{document}

