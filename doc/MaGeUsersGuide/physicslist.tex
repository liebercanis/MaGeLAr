\section{Introduction}

This section gives a high-level description of the physics list used
in the MaGe framework, and documents the MGProcesses Messenger that
provides a small amount of user control over that list. All macro commands 
relative to physics list must be issued in the \texttt{Pre\_init} phase, 
namely before the \texttt{/run/initialize}. 
 
\section{MGProcesses description}
The MaGe framework physics list is an amalamation of two physics lists
that are distributed with the GEANT4 simulation package. One part of
the list comes from the list included as part of the Underground
Physics advanced example. The other part of the list comes from the
QGSP\_HP hadronic list. Documentation on these lists are available on
the GEANT4 website. It manages the processes for Optical Photons 
(Cherenkov emission, Scintillation, Absorption) supports the cut-per-region 
approach: this allows to set different production cuts in different regions 
of the geometry.

The primary class is the MGProcessesList class, but it uses
supporting classes MGProcessesMaxTimeCuts, MGProcessesMinEkineCuts,
and MGProcessesSpecialCuts.

\section{Details of the physics lists}
\subsection{Interactions of hadrons}
The hadronic part of the physics list (default) is composed by 
\begin{enumerate}
\item theory-driven quark-gluon string models (QGSP) for the energetic pions, kaons, and nucleons, 
up to 100~TeV; 
\item LEP parametrized models for inelastic interactions of pions and nucleons below 25~GeV;
\item tabulated cross-section data derived from the 
ENDF/B-VI database to model capture, fission elastic scattering and 
inelastic scattering of neutrons from thermal energies up to 20~MeV (HP models). 
\end{enumerate}
%
Alternatively, the physics list developed by M.~Bauer (QGSP\_GN\_HP\_BIC\_ISO) for the description 
of muon-induced neutron production can be used. The list is chosen using the macro command
\begin{lstlisting}
 /MG/processes/qgsp_hadron_list true
\end{lstlisting}
The main difference with respect to the default physics list is that Binary cascade models (BIC) 
are used insted of LEP parametrized models for the 
description of nucleon and pion interactions below 10~GeV. It is also possible to replace the 
Binary cascade models with the Bertini (BERT) cascade models for the same energy range. 
The macro command is
\begin{lstlisting}
 /MG/processes/useBertCascade true
\end{lstlisting}
If the physics list is set to M.~Bauer's one , the quark-gluon plasma string 
model (QGSP) 
for the description of high-energy interactions can be replaced with chiral-invariant-phase-space 
models (CHIPS) for the de-excitation for the quark-gluon plasma. The macro command is 
\begin{lstlisting}
 /MG/processes/useQGSC true
\end{lstlisting}
As specified above, the latter command works only after
\begin{lstlisting}
 /MG/processes/qgsp_hadron_list true
\end{lstlisting}
%
For some applications, it may be useful to switch off all the hadronic interactions. It is 
possible in MaGe with the command 
\begin{lstlisting}
 /MG/processes/useNoHadPhysics true
\end{lstlisting}
In this case, all the additional specifiers described avove (e.g. useBertCascade or useQGSC) have no 
effect.
%
\subsection{Interactions of electrons, positrons and $\gamma$-rays}
The specific low-energy \geant\ models~\cite{geant4physics} are used by default for the description 
of the electromagnetic interactions of $\gamma$-rays, electrons, positrons and ions. The 
low-energy models include atomic effects (fluorescence, Doppler broadening) and can handle 
interactions down to 250~eV. The processes registered for $\gamma$-rays are coherent 
scattering (a.k.a. Rayleigh scattering), Compton scattering, photo-electric effect and pair 
production. Electromagnetic processes registered for electrons and positrons are ionisation, 
bremsstrahlung, $e^{+}$-annihilation and synchrotron radiation. Specific low-energy models 
are not available for $e^{+}$ (only for $e^{-}$); standard \geant\ models (see below) are 
used for the positron electromagnetic interactions. \\
Alternatively, electromagnetic interactions in \mage\ can be treated using the 
so-called Standard models provided by \geant. These models are tailored for high-energy 
applications; they are less precise in the low-energy part, not including atomic effect, but are 
faster from the point of view of CPU-time. The choice between Low-energy (default) and Standard 
electromagnetic models is done at run-time by the macro command
\begin{lstlisting}
 /MG/processes/lowenergy true/false
\end{lstlisting}
(true is the default). \\
Interactions of $\gamma$-rays and $e^{\pm}$ with hadrons are described by theory-driven quark-gluon 
string models for energetic particles ($> 3.5$~GeV) and by chiral invariant phase space (CHIPS) 
models for lower energies.
%
\subsection{Interactions of muons}
Hadronic interactions of high-energy muons are simulated using the \texttt{G4MuNuclearInteraction} model, 
replacing the \texttt{G4MuonNucleus} model. Although \texttt{G4MuNuclearInteraction} is known to be in 
closer agreement with experimental data, in old versions of \geant\ the processes had a bug in the 
desctructor, causing a crash of the program at the exit. While this bug had no effect on the simulation 
results, it prevented the clean exit from the program. \\
For those application not requiring the precise description of muon-induced effects (e.g. studies 
of radioactive background), the \texttt{G4MuonNucleus} can be registered to the physics list 
instead of \texttt{G4MuNuclearInteraction}, giving the command 
\begin{lstlisting}
 /MG/processes/useMuonNucleusProc true
\end{lstlisting}
The advantage is that one always has clean exit from the \mage\ runs.\\
Standard models only are available in \geant\ for electromagnetic interactions of muons (ionization, 
bremsstrahlung and pair production). 
%
\subsection{Interactions of optical photons}
\geant\ is able to handle optical photons and \mage\ takes advantage from such capability. While 
the default \mage\ physics list does \emph{not} include interactions of optical photons, these 
processes can be switched on with the command
\begin{lstlisting}
 /MG/processes/optical true/false
\end{lstlisting}
(false is the default). The models included in \mage\ encompass scintillation light 
emission (possibly with different light yield for electrons, $\alpha$-particles and nuclei), 
Cherenkov light emission, absorption, boundary processes, Rayleigh scattering and wavelength shifting. 
In order to produce correct results for optical photons, it is indeed necessary to specify in the 
geometry definition all the relevant optical properties of interfaces and bulk materials 
(scintillation yield, refraction index, absorption length, etc.). An example of this can be 
found in the geometry \texttt{GEMPIKLArGe.cc} (\texttt{munichteststand} directory). 
%
\subsection{Cuts definition}
As a general feature, \geant\ has not tracking cut, but only production cuts for 
$\delta$-rays and for soft bremsstrahlung photons. Cuts are expressed in range, and they are 
internally converted in energy threshold for each material used in the geometry. \\
In most applications, it is necessary to find a trade-off between accuracy and computing time. For this 
reason \mage\ provides three simulation ``realms'', with different production cuts. The realms 
are called \textsc{DarkMatter}, \textsc{BBDecay} and \textsc{CosmicRays}. \\ 
\begin{itemize}
\item \textsc{DarkMatter} realm is used for high precision simulations, especially related to background 
studies for dark matter applications: cut-in-range for $\gamma$-rays and e$^{\pm}$ are 5~$\mu$m and 
0.5~$\mu$m, respectively, corresponding to 1-keV threshold in metallic germanium. 
\item \textsc{BBdecay}  realm (default) is suitable for background studies related to double-beta 
decay, i.e. in the MeV region: the cut-in-range is tuned to give a 100-keV threshold for $\delta$-rays 
in metallic germanium (it is 0.1~mm for e$^{\pm}$ and $\gamma$-rays). 
\item  \textsc{CosmicRay} realm is used for the simulation of extensive electromagnetic showers 
induced by cosmic ray muons. The cuts are different in the volumes belonging to to the 
\texttt{Sensitive Region} and the other volumes. Cuts for the volumes in the sensitive region are the 
same than for the double-beta realm, while they are more relaxed everywhere else (5~cm for 
$\gamma$-rays and 1~cm for e$^{\pm}$), in order to save CPU time  and to avoid the precise tracking 
of particles in the uninteresting parts of the setups). Such a capability is based on the general 
cut-per-region feature of \geant. Sensitive volumes are registered in the geometry definition using the 
code reported in \ref{macros-processes}. 
\end{itemize}
The simulation realm is selected by macro using the command 
\begin{lstlisting}
/MG/processes/realm BBdecay/DarkMatter/CosmicRays
\end{lstlisting}
(\textsc{BBdecay} is the default). The 
simulation realm can be changed at run-time, even after the run initialization. 
