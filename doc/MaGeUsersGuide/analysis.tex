\section{Handling of decay chains} 

Radioactive decay in \geant \ is handled by the radioactive decay module(RDM).
The RDM will handle an entire decay chain, including proper branching ratios.
The chain all happens in one \geant \ event, so it is up to the user to divvy up
the chain into events that a physical detector would actually measure.  One way
of doing this is to keep track of each particle's global time.  The global time
of a track in \geant \ is the time since the track was created({\tt
aTrack$\rightarrow$GetGlobalTime()}).   Steps of an event that fall within a
user-prescribed window of time are then gathered together.  The time window
would typically correspond to the time resolution of the DAQ.  The user's output
class would then write to a \rootv \ object more often than once a \geant \ event.  

A problem arises because of the finite precision of a double.  In decimal form,
a G4double is only accurate to 15 digits or so.  Following the decay of
$^{238}$U, the global time of the daughter nucleus, $^{234}$Th would be on the
order of 5 $\times 10^9$ years, or $10^{17}$ seconds.  Each additional decay
only adds more to the global time.  If one wishes to gather steps that fall
within a typical time window of say, 15 $\mu$s, then one must find steps that
differ in time by $\sim$ 1 part in $10^{22}$.  This is treated as zero.

The workaround to this involves resetting the GlobalTime of a track that was
spawned by radioactive decay to zero if it is too large.  Any comparison of
global times can then yield the desired sensitivity.  A new variable, {\tt
fOffsetTime} is kept separate from GlobalTime, and is incremented to keep track
of the actual global time.

\subsection{Stacking Actions}

A brief comment on stacking actions.  When a track is created by a process, it
is sent to any UserStackingAction that is in place.  In \mage, the
UserStackingAction resides in the output classes that derive from {\tt
MGVOutputManager}.  The StackingAction can look at the track and determine
whether to put it in the ``Urgent" stack or the ``Waiting" stack.  When all
tracks in the urgent stack are finished, then all tracks in the waiting stacks
are transferred(all at once) to the urgent stack.  The urgent stack is then
processed as normal by \geant.  The StackingAction allows for a user-defined
{\tt NewStage()} function that is called once all tracks in the urgent stack are
finished.

\subsection{TimeWindower Implementation}

The UserStackingAction set in MaGe.cc is {\tt MGManagementStackingAction}.  It
has three functions, {\tt ClassifyNewTrack(const *G4Track)}, {\tt NewStage()},
and {\tt PrepareNewEvent()}.  These functions call their corresponding
counterparts in whatever output class you use that derives from {\tt
MGVOutputManager}.  One note of possible confusion, the {\tt
ClassifyNewTrack(const *G4Track)} points to output function {\tt
StackingAction(const *G4Track)} in the output class.  The {\tt
StackingAction(const *G4Track)} and the {\tt NewStage()} functions are written
in {\tt MGVOutputManager} and probably don't need to be messed with unless
you're doing something complicated.  The {\tt MGVOutputManager} also defines a
{\tt WritePartialEvent(const *G4Event)} and {\tt ResetPartialEvent(const
*G4Event)} functions.  Each individual output class needs to have its own
version of these two functions to properly use the TimeWindower.

The use of time windowing is specified by macro(default is off), as is the actual window of time to use.
If the macro specifies time windowing, then the {\tt StackingAction(const
*G4Track)} looks at each track as it is created.  If the creator process was
RadioactiveDecay and the GlobalTime is larger than the time window, then:

\begin{itemize}
\item the track's global time is stored
\item the track's global time is set to zero
\item the track is given the ``waiting" classification, so it is sent to the waiting stack
\end{itemize}

If all tracks in the urgent stack have been processed and there are tracks in
the waiting stack, then {\tt NewStage()} is called.  The {\tt fOffsetTime} is
updated, and {\tt WritePartialEvent(const *G4Event)} and {\tt
ResetPartialEvent(const *G4Event)} are called.  These functions obviously
depends on the output class being used, but the general idea is that the event
is written to a \rootv \ file, and any necessary arrays/variables/etc... are reset.
Only after all this are the tracks in the waiting stack sent to the urgent
stack.  It's worth mentioning that {\tt NewStage()} only happens if there are
tracks in the waiting stack.  If not, then the output class's {\tt
EndOfEventAction()} should handle the end of the actual \geant \ event.


\subsection{How to use the TimeWindower}

First, make sure that the WritePartialEvent() and ResetPartialEvent() functions have been implemented in your 
output class(look at the {\tt MGOutputG4Steps} class for an example of how to do
this).  You will also need to add messenger commands to your output messenger
class.  You can handle everything by messenger in a macro.  The macro commands
are:

{\tt
/MG/io/G4Steps/useTimeWindow

/MG/io/G4Steps/setTimeWindow
}

\noindent Where you would substitute your output class for G4Steps.

%\end{document}  



