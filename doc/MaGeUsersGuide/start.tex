% $Id: start.tex,v 1.3 2007-10-18 11:53:27 mjelen Exp $

%%% Local Variables:
%%% TeX-master: "usersguide"
%%% End:
\label{chapter:start}


\section{The Gerda README: Getting to know Macros}
\label{sec:README}

\mage is a simulation package based on \geant. It can be run using macro files (see chapter \ref{chapter:macros}).

A lot of different macro files exist for various purposes ranging from drawing a test stand setup to simulating the background that is to be expected in the final setup of the \gerda \ experiment.

A README file was created to support new users in getting comfortable with the usage of the \gerda \ macros.

The \gerda \ README file can be found in the \gerda \ CVS repository under 
\begin{lstlisting}
/MaGe/macros/Gerda/README
\end{lstlisting}
An introduction gives a short overview on visualisation drivers used for different purposes, e.g. for presentations or for applications which require interactive features.
The macros used in the \mage \ simulations for the \gerda \ experiment are listed alphabetically and their purpose, the geometrical setup and the expected output are described in a few sentences.


 
